The far-infrared (IR) frequency range, or more specifically the terahertz (THz) frequency range, of the electromagnetic spectrum has been comparatively underutilised in the field of solid-state vibrational spectroscopy. This was historically owing to a lack of coherent sources and detectors but now stems from the challenges involved with the interpretation of the resultant spectra as the excited motions in this frequency range consist of complex motions that must be deciphered with the aid of computational simulation using methods such as Density Functional Theory (DFT). This, in turn, gives rise to its own challenges as the accurate simulation of large organic molecular crystals can be computationally expensive. 

In DFT, weak intermolecular bonds such as H-bonds are poorly represented and an empirical correction is often included to account for them. These are called dispersion corrections and an investigation into the appropriate dispersion correction for \(\alpha\)-Lactose Monohydrate (\(\alpha\)-LM) was conducted. This molecule was chosen owing to its uncommonly sharp absorption peaks in this frequency range. DFT uses the harmonic approximation to calculate vibrational mode frequencies but this will inevitably remove important anharmonic effects, such as thermal expansion, from the calculation which may reduce correlation between calculated and experimental spectra. The Quasi-Harmonic Approximation (QHA) allows for thermal expansion to be incorporated into the system by applying the harmonic approximation to a range of volumes. This was used to calculate the thermodynamic properties and temperature dependent mode properties of \(\alpha\)-LM. This in turn was used to calculate the THz absorption spectrum of \(\alpha\)-LM at \SI{300}{K}.

The primary source of THz radiation for broadband spectroscopy is photoconductive switches. These are excited with femtosecond laser pulses that are focused onto a gap between two electrodes. An investigation into the appropriate gap-size for a photoconductive switch being excited by a \SI{150}{mW} fibre laser was carried out, with the appropriate gap-size being determined to be \SI{20}{\micro\metre}.
