The COVID pandemic had an extremely significant effect on my PhD. Access to the laboratory was removed completely for approximately 6 months. This was longer than the mandated closures across the country owing to the issues with the universities water supply and possible Legionella bacterial infection which required an additional two months before re\nobreakdash-opening.  Once reopened, strict occupancy rules for the laboratory and competing demands for shared laboratory space by of other members of the group continued to reduce my available time in the laboratory. This led to a significant restructuring of my project to have a much larger computational component than was originally planned. Laboratory time was further reduced owing to positive COVID tests and associated quarantine restrictions. Attempts to grow a single crystal of \(\alpha\)-Lactose Monohydrate were stopped when the University began lock\nobreakdash-downs and were not resumed owing to limited availability of appropriate laboratory space that could be regularly accessed owing to occupancy rules.  Once occupancy rules were lifted it was deemed too late to continue with this aspect of the project. This drastically reduced the ability to interpret some of the calculated thermodynamic properties presented in Chapter~5 without corresponding variable X-ray and calorimetric measurements of single crystals.

Additionally, the supply chain issues that arose from the pandemic affected the cost of liquid He, which is the primary coolant used to take low temperature THz-TDS spectra. Owing to the significantly increased cost, reduced availability and greater need of other members of the group for what He was available, no further low temperature spectra were produced. Ideally, all experimental spectra shown in this work would have been reproduced to improve their quality, increasing SNR and reducing etalons in the the measurements that can at times obscure some of the smaller spectral features. This also would have included measuring the spectrum at more temperatures than are presented in Chapter~5, which would have allowed for a more thorough analysis of the temperature response of the system. 

Both the initial COVID shutdown and then additional supply chain issues also significantly affected the repair time for the fibre laser for System~3 which was presented in Chapter~6. The fibre laser was to be the focus of a new Attenuated-total-reflection (ATR) based spectrometer which would have used the photoconductive emitters demonstrated in Chapter~6. While initial testing was performed on a range of other laser systems when it came to develop the fibre-laser based system in February 2020 initial measurements proved challenging and we eventually found a fault with the laser that was causing significant power instability. This resulted in the laser requiring factory repair and this was finally up and running again in November 2020. COVID shutdowns and the move of the Leeds Cleanroom to the new building also limited the number of photoconductive devices available for this project. These delays meant that while characterisation of the various devices could be performed, the final setup of the ATR system could not be completed in a realistic time-frame. The final aim was to use this system to measure spectra of Lactose and use the effective medium approximation for ATR spectra already included in PDielec in combination with the calculated results already presented in chapter~4 and 5. 

Finally, owing to the length of and significant isolation caused by the lock\nobreakdash-downs, stress as a result of the problems described above and a number of personal issues, the COVID pandemic had a drastic effect on my mental health for which I had to seek professional help. These problems lasted for approximately 18 months from the beginning of the pandemic.